% Options for packages loaded elsewhere
\PassOptionsToPackage{unicode}{hyperref}
\PassOptionsToPackage{hyphens}{url}
%
\documentclass[
  12pt,
]{article}
\usepackage[]{mathpazo}
\usepackage{amssymb,amsmath}
\usepackage{ifxetex,ifluatex}
\ifnum 0\ifxetex 1\fi\ifluatex 1\fi=0 % if pdftex
  \usepackage[T1]{fontenc}
  \usepackage[utf8]{inputenc}
  \usepackage{textcomp} % provide euro and other symbols
\else % if luatex or xetex
  \usepackage{unicode-math}
  \defaultfontfeatures{Scale=MatchLowercase}
  \defaultfontfeatures[\rmfamily]{Ligatures=TeX,Scale=1}
\fi
% Use upquote if available, for straight quotes in verbatim environments
\IfFileExists{upquote.sty}{\usepackage{upquote}}{}
\IfFileExists{microtype.sty}{% use microtype if available
  \usepackage[]{microtype}
  \UseMicrotypeSet[protrusion]{basicmath} % disable protrusion for tt fonts
}{}
\makeatletter
\@ifundefined{KOMAClassName}{% if non-KOMA class
  \IfFileExists{parskip.sty}{%
    \usepackage{parskip}
  }{% else
    \setlength{\parindent}{0pt}
    \setlength{\parskip}{6pt plus 2pt minus 1pt}}
}{% if KOMA class
  \KOMAoptions{parskip=half}}
\makeatother
\usepackage{xcolor}
\IfFileExists{xurl.sty}{\usepackage{xurl}}{} % add URL line breaks if available
\IfFileExists{bookmark.sty}{\usepackage{bookmark}}{\usepackage{hyperref}}
\hypersetup{
  pdftitle={Tentative Migration Reading List},
  pdfauthor={Andrea Miranda-Gonzalez},
  hidelinks,
  pdfcreator={LaTeX via pandoc}}
\urlstyle{same} % disable monospaced font for URLs
\usepackage[margin=1in]{geometry}
\usepackage{graphicx,grffile}
\makeatletter
\def\maxwidth{\ifdim\Gin@nat@width>\linewidth\linewidth\else\Gin@nat@width\fi}
\def\maxheight{\ifdim\Gin@nat@height>\textheight\textheight\else\Gin@nat@height\fi}
\makeatother
% Scale images if necessary, so that they will not overflow the page
% margins by default, and it is still possible to overwrite the defaults
% using explicit options in \includegraphics[width, height, ...]{}
\setkeys{Gin}{width=\maxwidth,height=\maxheight,keepaspectratio}
% Set default figure placement to htbp
\makeatletter
\def\fps@figure{htbp}
\makeatother
\setlength{\emergencystretch}{3em} % prevent overfull lines
\providecommand{\tightlist}{%
  \setlength{\itemsep}{0pt}\setlength{\parskip}{0pt}}
\setcounter{secnumdepth}{-\maxdimen} % remove section numbering
\linespread{1.05}

\title{Tentative Migration Reading List}
\author{Andrea Miranda-Gonzalez}
\date{September 2020}

\begin{document}
\maketitle

{
\setcounter{tocdepth}{2}
\tableofcontents
}
\hypertarget{why-people-migrate}{%
\section{\texorpdfstring{\textbf{Why people
migrate?}}{Why people migrate?}}\label{why-people-migrate}}

\hypertarget{early-economic-theories}{%
\subsection{\texorpdfstring{\emph{Early economic
theories}}{Early economic theories}}\label{early-economic-theories}}

Chiswick, Barry (1999). ``Are immigrants favorably self-selected?'' In:
\emph{American Economic Review} 89.2, pp.~181--185.

Borjas, George J (1989). ``Economic theory and international
migration''. In: \emph{International migration review} 23.3,
pp.~457--485.

Stark, Oded and J Edward Taylor (1989). ``Relative deprivation and
international migration oded stark''. In: \emph{Demography} 26.1,
pp.~1--14.

Stark, Oded and David E Bloom (1985). ``The new economics of labor
migration''. In: \emph{The american Economic review} 75.2, pp.~173--178.

Zelinsky, Wilbur (1971). ``The hypothesis of the mobility transition''.
In: \emph{Geographical review}, pp.~219--249.

Harris, John R and Michael P Todaro (1970). ``Migration, unemployment
and development: a two-sector analysis''. In:
\emph{The American economic review} 60.1, pp.~126--142.

Todaro, Michael P (1969). ``A model of labor migration and urban
unemployment in less developed countries''. In:
\emph{The American economic review} 59.1, pp.~138--148.

Lee, Everett S (1966). ``A theory of migration''. In: \emph{Demography}
3.1, pp.~47--57.

Sjaastad, Larry A (1962). ``The costs and returns of human migration''.
In: \emph{Journal of political Economy} 70.5, Part 2, pp.~80--93.

Ravenstein, Ernest George (1889). ``The laws of migration''. In:
\emph{Journal of the royal statistical society} 52.2, pp.~241--305.

\hypertarget{contemporary-sociological-theories}{%
\subsection{\texorpdfstring{\emph{Contemporary sociological
theories}}{Contemporary sociological theories}}\label{contemporary-sociological-theories}}

Garip, Filiz (2019).
\emph{On the move: Changing mechanisms of Mexico-US migration}. Vol. 2.
Princeton University Press.

Ryo, Emily (2013). ``Deciding to cross: norms and economics of
unauthorized migration''. In: \emph{American Sociological Review} 78.4,
pp.~574--603.

Williams, Allan M and Vladimir Baláž (2012). ``Migration, risk, and
uncertainty: Theoretical perspectives''. In:
\emph{Population, Space and Place} 18.2, pp.~167--180.

Hatton, Timothy J, Jeffrey G Williamson, and others (2005). ``Global
migration and the world economy''. In: \emph{MIT Press, Cambridge, MA}
289, p.~12.

McKeown, Adam (2004). ``Global Migration, 1846-1940''. In:
\emph{Journal of World History}, pp.~155--189.

Massey, Douglas S. (1999). ``Why Does Immigration Occur?: A Theoretical
Synthesis''. In:
\emph{Handbook of International Migration, The: The American Experience}.
Russell Sage Foundation, pp.~34--52. ISBN: 9780871542441.
\url{http://www.jstor.org/stable/10.7758/9781610442893.9}.

Massey, Douglas S, Joaquin Arango, Graeme Hugo, Ali Kouaouci, Adela
Pellegrino, and J Edward Taylor (1993). ``Theories of international
migration: A review and appraisal''. In:
\emph{Population and development review}, pp.~431--466.

Sassen, Saskia (1989). ``America's Immigration" Problem"''. In:
\emph{World Policy Journal} 6.4, pp.~811--832.

Piore, Michael J (1979). ``Birds of passage: Migrant labor and
industrial societies, 1979''. In: \emph{Cambridge: CUP}.

\hypertarget{role-of-the-state}{%
\subsection{\texorpdfstring{\emph{Role of the
State}}{Role of the State}}\label{role-of-the-state}}

Hollifield, James F, Philip L Martin, and Pia M Orrenius (2014). ``The
dilemmas of immigration control''. In:
\emph{Controlling immigration: A global perspective}, pp.~3--34.

Ruhs, Martin and Philip Martin (2008). ``Numbers vs.~rights: trade-offs
and guest worker programs''. In: \emph{International Migration Review}
42.1, pp.~249--265.

Boswell, Christina (2007). ``Theorizing migration policy: Is there a
third way?'' In: \emph{International migration review} 41.1,
pp.~75--100.

Cornelius, Wayne A (2005). ``Controlling ‘unwanted’immigration:
Lessons from the United States, 1993--2004''. In:
\emph{Journal of Ethnic and Migration Studies} 31.4, pp.~775--794.

Neumayer, Eric (2005). ``Bogus refugees? The determinants of asylum
migration to Western Europe''. In:
\emph{International studies quarterly} 49.3, pp.~389--409.

Zolberg, Aristide R (1999). ``Matters of State: Theorizing Immigration
Policy''. In:
\emph{Handbook of International Migration: The American Experience}. Ed.
by Philip Kasinitz Charles Hirschman and Joseph DeWind. New York:
Russell Sage Foundation, pp.~71--93.

Torpey, John (1998). ``Coming and going: On the state monopolization of
the legitimate “means of movement”''. In: \emph{Sociological theory}
16.3, pp.~239--259.

Freeman, Gary P (1995). ``Modes of immigration politics in liberal
democratic states''. In: \emph{International migration review} 29.4,
pp.~881--902.

\begin{center}\rule{0.5\linewidth}{0.5pt}\end{center}

\hypertarget{how-do-we-observe-migration-data-and-measurements}{%
\section{\texorpdfstring{\textbf{How do we observe migration? Data and
measurements}}{How do we observe migration? Data and measurements}}\label{how-do-we-observe-migration-data-and-measurements}}

\hypertarget{data-sources}{%
\subsection{\texorpdfstring{\emph{Data
sources}}{Data sources}}\label{data-sources}}

Cesare, Nina, Hedwig Lee, Tyler McCormick, Emma Spiro, and Emilio
Zagheni (2018). ``Promises and pitfalls of using digital traces for
demographic research''. In: \emph{Demography} 55.5, pp.~1979--1999.

Zagheni, Emilio, Ingmar Weber, and Krishna Gummadi (2017). ``Leveraging
facebook's advertising platform to monitor stocks of migrants''. In:
\emph{Population and Development Review}, pp.~721--734.

Hughes, Christina, Emilio Zagheni, Guy J Abel, Alessandro Sorichetta,
Arkadius Wi'sniowski, Ingmar Weber, and Andrew J Tatem (2016).
``Inferring Migrations: Traditional Methods and New Approaches based on
Mobile Phone, Social Media, and other Big Data: Feasibility study on
Inferring (labour) mobility and migration in the European Union from big
data and social media data''. In:
\emph{Report for the European Commission}.

Messias, Johnnatan, Fabricio Benevenuto, Ingmar Weber, and Emilio
Zagheni (2016). ``From migration corridors to clusters: The value of
Google+ data for migration studies''. In:
\emph{2016 IEEE/ACM International Conference on Advances in Social Networks Analysis and Mining (ASONAM)}.
IEEE. , pp.~421--428.

Weber, Ingmar (2015). ``Demographic research with non-representative
internet data''. In: \emph{International Journal of Manpower} 36.1,
pp.~13--25.

Beauchemin, Cris (2014). ``A manifesto for quantitative multi-sited
approaches to international migration''. In:
\emph{International Migration Review} 48.4, pp.~921--938.

Hawelka, Bartosz, Izabela Sitko, Euro Beinat, Stanislav Sobolevsky,
Pavlos Kazakopoulos, and Carlo Ratti (2014). ``Geo-located Twitter as
proxy for global mobility patterns''. In:
\emph{Cartography and Geographic Information Science} 41.3,
pp.~260--271.

Moed, Henk F and Gali Halevi (2014). ``A bibliometric approach to
tracking international scientific migration''. In: \emph{Scientometrics}
101.3, pp.~1987--2001.

Ruggles, Steven (2014). ``Big microdata for population research''. In:
\emph{Demography} 51.1, pp.~287--297.

Zagheni, Emilio, Venkata Rama Kiran Garimella, Ingmar Weber, and Bogdan
State (2014). ``Inferring international and internal migration patterns
from twitter data''. In:
\emph{Proceedings of the 23rd International Conference on World Wide Web}.
, pp.~439--444.

Blumenstock, Joshua E (2012). ``Inferring patterns of internal migration
from mobile phone call records: evidence from Rwanda''. In:
\emph{Information Technology for Development} 18.2, pp.~107--125.

Kaplan, Greg and Sam Schulhofer-Wohl (2012). ``Interstate migration has
fallen less than you think: Consequences of hot deck imputation in the
Current Population Survey''. In: \emph{Demography} 49.3, pp.~1061--1074.

Bycroft, CHRISTINE (2006). ``Challenges in estimating populations''. In:
\emph{New Zealand Population Review} 32.2, pp.~21--47.

Van Hook, Jennifer, Weiwei Zhang, Frank D Bean, and Jeffrey S Passel
(2006). ``Foreign-born emigration: A new approach and estimates based on
matched CPS files''. In: \emph{Demography} 43.2, pp.~361--382.

Dumont, Jean-Christophe and Georges Lemaître (2005). ``Counting
immigrants and expatriates in OECD countries: a new perspective''. In:
\emph{OECD Social, Employment and Migration Working Papers, No. 25, OECD Publishing}.

Bell, Martin, Philip Rees, and Tom Wilson (2003). ``Comparing internal
migration between countries: who collects what?'' In:
\emph{Discussion Paper}.

Rees, Philip, Martin Bell, Oliver Duke-Williams, and Marcus Blake
(2000). ``Problems and solutions in the measurement of migration
intensities: Australia and Britain compared''. In:
\emph{Population Studies} 54.2, pp.~207--222.

Massey, Douglas S and Audrey Singer (1995). ``New estimates of
undocumented Mexican migration and the probability of apprehension''.
In: \emph{Demography} 32.2, pp.~203--213.

Schmertmann, Carl P (1992). ``Estimation of historical migration rates
from a single census: Interregional migration in Brazil 1900--1980''.
In: \emph{Population Studies} 46.1, pp.~103--120.

Nicholson, Beryl (1990). ``The hidden component in census-derived
migration data: assessing its size and distribution''. In:
\emph{Demography} 27.1, pp.~111--119.

Skeldon, Ronald and others (1990).
\emph{Population mobility in developing countries.} Belhaven Press.

Hill, Kenneth (1987). ``New approaches to the estimation of migration
flows from census and administrative data sources''. In:
\emph{International Migration Review} 21.4, pp.~1279--1303.

Clark, William AV and others (1985). ``Human migration''. In:
\emph{Regional Research Institute, West Virginia University Book Chapters},
pp.~1--51.

Shryock, Henry S, Jacob S Siegel, and Elizabeth A Larmon (1973).
\emph{The methods and materials of demography}. Vol. 2. US Bureau of the
Census.

\hypertarget{methods}{%
\subsection{\texorpdfstring{\emph{Methods}}{Methods}}\label{methods}}

Bernard, Aude (2017). ``Cohort measures of internal migration:
Understanding long-term trends''. In: \emph{Demography} 54.6,
pp.~2201--2221.

Rees, Philip, Martin Bell, Marek Kupiszewski, Dorota Kupiszewska,
Philipp Ueffing, Aude Bernard, Elin Charles-Edwards, and John Stillwell
(2017). ``The impact of internal migration on population redistribution:
An international comparison''. In: \emph{Population, Space and Place}
23.6, p.~e2036.

Klabunde, Anna and Frans Willekens (2016). ``Decision-making in
agent-based models of migration: state of the art and challenges''. In:
\emph{European Journal of Population} 32.1, pp.~73--97.

Liu, Mao-Mei, Mathew J Creighton, Fernando Riosmena, and Pau Baiazn
Mun͂oz (2016). ``Prospects for the comparative study of international
migration using quasi-longitudinal micro-data''. In:
\emph{Demographic research} 35, p.~745.

Riosmena, Fernando (2016). ``The potential and limitations of
cross-context comparative research on migration''. In:
\emph{The ANNALS of the American Academy of Political and Social Science}
666.1, pp.~28--45.

Bell, Martin (2015). ``Demography, time and space''. In:
\emph{Journal of Population Research} 32.3-4, pp.~173--186.

Bell, Martin, Elin Charles-Edwards, Dorota Kupiszewska, Marek
Kupiszewski, John Stillwell, and Yu Zhu (2015). ``Internal migration
data around the world: Assessing contemporary practice''. In:
\emph{Population, Space and Place} 21.1, pp.~1--17.

Bernard, Aude, Martin Bell, and Elin Charles-Edwards (2014).
``Life-course transitions and the age profile of internal migration''.
In: \emph{Population and Development Review} 40.2, pp.~213--239.

Wachter, Kenneth W (2014). \emph{Essential demographic methods}. Harvard
University Press.

Yusuf, Farhat, Jo M Martins, David A Swanson, Jo M Martins, and David A
Swanson (2014). \emph{Methods of demographic analysis}. Springer.

Bijak, Jakub (2010).
\emph{Forecasting international migration in Europe: A Bayesian view}.
Vol. 24. Springer Science \& Business Media.

Rogers, Andrei, Jani Little, and James Raymer (2010).
\emph{The indirect estimation of migration: Methods for dealing with irregular, inadequate, and missing data}.
Vol. 26. Springer Science \& Business Media.

-------- (2005). ``Bayesian methods in international migration
forecasting''. In: \emph{International Migration in Europe: Data}.
Central European Forum for Migration Research.

Bell, Martin, Marcus Blake, Paul Boyle, Oliver Duke-Williams, Philip
Rees, John Stillwell, and Graeme Hugo (2002). ``Cross-national
comparison of internal migration: issues and measures''. In:
\emph{Journal of the Royal Statistical Society: Series A (Statistics in Society)}
165.3, pp.~435--464.

Rees, Philip, Martin Bell, Oliver Duke-Williams, and Marcus Blake
(2000). ``Problems and solutions in the measurement of migration
intensities: Australia and Britain compared''. In:
\emph{Population Studies} 54.2, pp.~207--222.

Rogers, Andrei and Luis J Castro (1981). ``Model migration schedules''.
In: \emph{IIASA Research Report}.

\hypertarget{comments-on-type-of-methods-quantitative-vs-qualitative-methods}{%
\subsection{\texorpdfstring{\emph{Comments on type of methods:
quantitative vs qualitative
methods}}{Comments on type of methods: quantitative vs qualitative methods}}\label{comments-on-type-of-methods-quantitative-vs-qualitative-methods}}

Willekens, Frans, Douglas Massey, James Raymer, and Cris Beauchemin
(2016). ``International migration under the microscope''. In:
\emph{Science} 352.6288, pp.~897--899.

Bloemraad, I (2007). ``Of puzzles and serendipity: doing cross-national,
mixed method immigration research''. In:
\emph{Researching Migration. Stories from the Field}, pp.~34--49.

Massey, Douglas S and René Zenteno (2000). ``A validation of the
ethnosurvey: The case of Mexico-US migration''. In:
\emph{International migration review} 34.3, pp.~766--793.

Zenteno, Rene M and Douglas S Massey (1999). ``Especificidad versus
representatividad: enfoques metodológicos en el estudio de la migración
mexicana hacia Estados Unidos''. In:
\emph{Estudios demográficos y urbanos}, pp.~75--116. Handbook of
research methods in migration

\begin{center}\rule{0.5\linewidth}{0.5pt}\end{center}

\hypertarget{demographic-processes-within-migration}{%
\section{\texorpdfstring{\textbf{Demographic processes within
migration}}{Demographic processes within migration}}\label{demographic-processes-within-migration}}

\hypertarget{migration-and-fertility}{%
\subsection{\texorpdfstring{\emph{Migration and
fertility}}{Migration and fertility}}\label{migration-and-fertility}}

Choi, Kate H and Robert D Mare (2012). ``International migration and
educational assortative mating in Mexico and the United States''. In:
\emph{Demography} 49.2, pp.~449--476.

Parrado, Emilio A (2011). ``How high is Hispanic/Mexican fertility in
the United States? Immigration and tempo considerations''. In:
\emph{Demography} 48.3, pp.~1059--1080.

Davis, Jason and David Lopez-Carr (2010). ``The effects of migrant
remittances on population--environment dynamics in migrant origin areas:
international migration, fertility, and consumption in highland
Guatemala''. In: \emph{Population and Environment} 32.2-3, pp.~216--237.

Kulu, Hill (2005). ``Migration and fertility: Competing hypotheses
re-examined''. In:
\emph{European Journal of Population/Revue européenne de Démographie}
21.1, pp.~51--87.

Friedlander, Dov (1983). ``Demographic responses and socioeconomic
structure: population processes in England and Wales in the nineteenth
century''. In: \emph{Demography} 20.3, pp.~249--272.

\hypertarget{migration-health-and-mortality}{%
\subsection{\texorpdfstring{\emph{Migration, health and
mortality}}{Migration, health and mortality}}\label{migration-health-and-mortality}}

Riosmena, Fernando, Randall Kuhn, and Warren C Jochem (2017).
``Explaining the immigrant health advantage: Self-selection and
protection in health-related factors among five major national-origin
immigrant groups in the United States''. In: \emph{Demography} 54.1,
pp.~175--200.

Palloni, Alberto and Elizabeth Arias (2004). ``Paradox lost: explaining
the Hispanic adult mortality advantage''. In: \emph{Demography} 41.3,
pp.~385--415.

\hypertarget{migration-aging-and-the-life-course}{%
\subsection{\texorpdfstring{\emph{Migration, aging and the life
course}}{Migration, aging and the life course}}\label{migration-aging-and-the-life-course}}

Vega, Alma and Karen Hirschman (2019). ``The reasons older immigrants in
the United States of America report for returning to Mexico''. In:
\emph{Ageing \& Society} 39.4, pp.~722--748.

Ruiz, John M, Patrick Steffen, and Timothy B Smith (2013). ``Hispanic
mortality paradox: a systematic review and meta-analysis of the
longitudinal literature''. In: \emph{American Journal of Public Health}
103.3, pp.~e52--e60.

Acevedo-Garcia, Dolores, Emma V Sanchez-Vaznaugh, Edna A
Viruell-Fuentes, and Joanna Almeida (2012). ``Integrating social
epidemiology into immigrant health research: a cross-national
framework''. In: \emph{Social science \& medicine} 75.12,
pp.~2060--2068.

Feliciano, Cynthia (2005). ``Educational selectivity in US immigration:
How do immigrants compare to those left behind?'' In: \emph{Demography}
42.1, pp.~131--152.

Litwak, Eugene and Charles F Longino Jr (1987). ``Migration patterns
among the elderly: A developmental perspective''. In:
\emph{The Gerontologist} 27.3, pp.~266--272.

\begin{center}\rule{0.5\linewidth}{0.5pt}\end{center}

\hypertarget{types-of-migration}{%
\section{\texorpdfstring{\textbf{Types of migration
}}{Types of migration }}\label{types-of-migration}}

\hypertarget{international-and-internal-migration}{%
\subsection{\texorpdfstring{\emph{International and internal
migration}}{International and internal migration}}\label{international-and-internal-migration}}

Johnson, Kenneth M, Katherine J Curtis, and David Egan-Robertson (2017).
``Frozen in place: net migration in sub-national areas of the United
States in the era of the great recession''. In:
\emph{Population and development review} 43.4, p.~599.

Flippen, Chenoa (2013). ``Relative deprivation and internal migration in
the United States: A comparison of black and white men''. In:
\emph{American Journal of Sociology} 118.5, pp.~1161--1198.

Hanson, Gordon H and Craig McIntosh (2010). ``The great Mexican
emigration''. In: \emph{the Review of Economics and Statistics} 92.4,
pp.~798--810.

Mayda, Anna Maria (2010). ``International migration: A panel data
analysis of the determinants of bilateral flows''. In:
\emph{Journal of Population Economics} 23.4, pp.~1249--1274.

Greenwood, Michael J (1997). ``Internal migration in developed
countries''. In: \emph{Handbook of population and family economics} 1,
pp.~647--720.

\hypertarget{urbanization}{%
\subsection{\texorpdfstring{\emph{Urbanization}}{Urbanization}}\label{urbanization}}

Hall, Matthew, Kyle Crowder, and Amy Spring (2015). ``Neighborhood
foreclosures, racial/ethnic transitions, and residential segregation''.
In: \emph{American sociological review} 80.3, pp.~526--549.

Clark, William AV (1991). ``Residential preferences and neighborhood
racial segregation: A test of the Schelling segregation model''. In:
\emph{Demography} 28.1, pp.~1--19.

Davis, Kingsley (1955). ``The origin and growth of urbanization in the
world''. In: \emph{American Journal of Sociology} 60.5, pp.~429--437.

\hypertarget{forced-migration}{%
\subsection{\texorpdfstring{\emph{Forced
migration}}{Forced migration}}\label{forced-migration}}

Schon, Justin (2019). ``Motivation and opportunity for conflict-induced
migration: An analysis of Syrian migration timing''. In:
\emph{Journal of Peace Research} 56.1, pp.~12--27.

Riosmena, Fernando, Raphael Nawrotzki, and Lori Hunter (2018). ``Climate
migration at the height and end of the Great Mexican Emigration Era''.
In: \emph{Population and development review} 44.3, p.~455.

Hamlin, Rebecca and Philip E Wolgin (2012). ``Symbolic politics and
policy feedback: The United Nations Protocol relating to the status of
refugees and American refugee policy in the Cold War''. In:
\emph{International Migration Review} 46.3, pp.~586--624.

Neumayer, Eric (2004). ``Asylum destination choice: what makes some West
European countries more attractive than others?'' In:
\emph{European Union Politics} 5.2, pp.~155--180.

Lee, Erika (2002). ``Enforcing the borders: Chinese exclusion along the
US borders with Canada and Mexico, 1882--1924''. In:
\emph{The Journal of American History} 89.1, pp.~54--86.

Nwokeji, G Ugo (2001). ``African conceptions of gender and the slave
traffic''. In: \emph{The William and Mary Quarterly} 58.1, pp.~47--68.

Horowitz, Michael M (1991). ``Victims upstream and down''. In:
\emph{Journal of Refugee Studies} 4.2, pp.~164--181.

Zolberg, Aristide R, Astri Suhrke, and Sergio Aguayo (1986).
``International factors in the formation of refugee movements''. In:
\emph{International migration review} 20.2, pp.~151--169.

\begin{center}\rule{0.5\linewidth}{0.5pt}\end{center}

\hypertarget{integrationassimilation}{%
\section{\texorpdfstring{\textbf{Integration/Assimilation}}{Integration/Assimilation}}\label{integrationassimilation}}

Lee, Jennifer and Min Zhou (2015).
\emph{The Asian American Achievement Paradox}. Russell Sage Foundation.

Alba, Richard, Tomás R Jiménez, and Helen B Marrow (2014). ``Mexican
Americans as a paradigm for contemporary intra-group heterogeneity''.
In: \emph{Ethnic and Racial Studies} 37.3, pp.~446--466.

Alba, Richard D (2009).
\emph{Remaking the American mainstream: Assimilation and contemporary immigration}.
Harvard University Press.

Kasinitz, Philip, John H Mollenkopf, Mary C Waters, and Jennifer
Holdaway (2009).
\emph{Inheriting the city: The children of immigrants come of age}.
Russell Sage Foundation.

Brown, Susan K and Frank D Bean (2006). ``Assimilation models, old and
new: Explaining a long-term process''. In:
\emph{Migration information source}.

Perlmann, Joel (2005).
\emph{Italians then, Mexicans now: Immigrant origins and the second-generation progress, 1890-2000}.
Russell Sage Foundation.

Gordon, Milton Myron (1964).
\emph{Assimilation in American life: The role of race, religion, and national origins}.
Oxford University Press on Demand.

\hypertarget{segmented-assimilation}{%
\subsection{\texorpdfstring{\emph{Segmented
assimilation}}{Segmented assimilation}}\label{segmented-assimilation}}

Soehl, Thomas, Roger Waldinger, and Renee Luthra (2020). ``Social
politics: the importance of the family for naturalisation decisions of
the 1.5 generation''. In: \emph{Journal of Ethnic and Migration Studies}
46.7, pp.~1240--1260.

Haller, William, Alejandro Portes, and Scott M Lynch (2011). ``Dreams
fulfilled, dreams shattered: Determinants of segmented assimilation in
the second generation''. In: \emph{Social forces} 89.3, pp.~733--762.

Xie, Yu and Emily Greenman (2011). ``The social context of assimilation:
Testing implications of segmented assimilation theory''. In:
\emph{Social science research} 40.3, pp.~965--984.

Portes, Alejandro, Patricia Fernández-Kelly, and William Haller (2009).
``The adaptation of the immigrant second generation in America: A
theoretical overview and recent evidence''. In:
\emph{Journal of ethnic and migration studies} 35.7, pp.~1077--1104.

Telles, Edward E and Vilma Ortiz (2008).
\emph{Generations of exclusion: Mexican-Americans, assimilation, and race}.
Russell Sage Foundation.

Portes, Alejandro and Min Zhou (1993). ``The new second generation:
Segmented assimilation and its variants''. In:
\emph{The annals of the American academy of political and social science}
530.1, pp.~74--96.

Gans, Herbert J (1992). ``Second-generation decline: Scenarios for the
economic and ethnic futures of the post-1965 American immigrants''. In:
\emph{Ethnic and racial studies} 15.2, pp.~173--192.

\hypertarget{role-of-legal-status}{%
\subsection{\texorpdfstring{\emph{Role of legal
status}}{Role of legal status}}\label{role-of-legal-status}}

Gonzales, Roberto G (2016).
\emph{Lives in limbo: Undocumented and coming of age in America}. Univ
of California Press.

Bean, Frank D, Susan K Brown, James D Bachmeier, Susan Brown, and James
Bachmeier (2015).
\emph{Parents without papers: The progress and pitfalls of Mexican American integration}.
Russell Sage Foundation.

Dreby, Joanna (2015).
\emph{Everyday illegal: When policies undermine immigrant families}.
University of California Press.

-------- (2012). ``The burden of deportation on children in Mexican
immigrant families''. In: \emph{Journal of Marriage and Family} 74.4,
pp.~829--845.

Menjívar, Cecilia and Leisy Abrego (2012). ``Legal violence: Immigration
law and the lives of Central American immigrants''. In:
\emph{American journal of sociology} 117.5, pp.~000--000.

Sigona, Nando (2012). ``‘I have too much baggage’: the impacts of
legal status on the social worlds of irregular migrants''. In:
\emph{Social Anthropology} 20.1, pp.~50--65.

Brown, Hana E (2011). ``Refugees, rights, and race: How legal status
shapes Liberian immigrants' relationship with the state''. In:
\emph{Social Problems} 58.1, pp.~144--163.

-------- (2011). ``Learning to be illegal: Undocumented youth and
shifting legal contexts in the transition to adulthood''. In:
\emph{American sociological review} 76.4, pp.~602--619.

Yoshikawa, Hirokazu (2011).
\emph{Immigrants raising citizens: Undocumented parents and their children}.
Russell Sage Foundation.

Gleeson, Shannon (2010). ``Labor rights for all? The role of
undocumented immigrant status for worker claims making''. In:
\emph{Law \& Social Inquiry} 35.3, pp.~561--602.

Cainkar, Louis A (2009).
\emph{Homeland insecurity: the Arab American and Muslim American experience after 9/11}.
Russell Sage Foundation.

Abrego, Leisy Janet (2006). ``“I can’t go to college because I
don’t have papers”: Incorporation patterns of Latino undocumented
youth''. In: \emph{Latino studies} 4.3, pp.~212--231.

Menjívar, Cecilia (2006). ``Liminal legality: Salvadoran and Guatemalan
immigrants' lives in the United States''. In:
\emph{American journal of sociology} 111.4, pp.~999--1037.

\hypertarget{undocumented-migration}{%
\subsection{\texorpdfstring{\emph{Undocumented
migration}}{Undocumented migration}}\label{undocumented-migration}}

Massey, Douglas S, Karen A Pren, and Jorge Durand (2016). ``Why border
enforcement backfired''. In: \emph{American journal of sociology} 121.5,
pp.~1557--1600.

Amuedo-Dorantes, Catalina and Susan Pozo (2014). ``On the intended and
unintended consequences of enhanced US border and interior immigration
enforcement: Evidence from Mexican deportees''. In: \emph{Demography}
51.6, pp.~2255--2279.

Donato, Katharine M and Amada Armenta (2011). ``What we know about
unauthorized migration''. In: \emph{Annual Review of Sociology} 37,
pp.~529--543.

De Genova, Nicholas (2004). ``The legal production of Mexican/migrant
“illegality”''. In: \emph{Latino studies} 2.2, pp.~160--185.

Ngai, Mae M (2003). ``The strange career of the illegal alien:
Immigration restriction and deportation policy in the United States,
1921--1965''. In: \emph{Law and History Review} 21.1, pp.~69--108.

De Genova, Nicholas P (2002). ``Migrant “illegality” and
deportability in everyday life''. In:
\emph{Annual review of anthropology} 31.1, pp.~419--447.

Nevins, Joseph and Allan Nevins (2002).
\emph{Operation gatekeeper: The rise of the" illegal alien" and the making of the US-Mexico boundary}.
Psychology Press. ***

\hypertarget{important-characteristic-of-migrants}{%
\section{\texorpdfstring{\textbf{Important characteristic of
migrants}}{Important characteristic of migrants}}\label{important-characteristic-of-migrants}}

\hypertarget{gender}{%
\subsection{\texorpdfstring{\emph{Gender}}{Gender}}\label{gender}}

Lutz, Helma (2010). ``Gender in the migratory process''. In:
\emph{Journal of ethnic and migration studies} 36.10, pp.~1647--1663.

Cantú, Lionel (2009).
\emph{The sexuality of migration: Border crossings and Mexican immigrant men}.
Vol. 5. NYU Press.

Acosta, Katie L (2008). ``Lesbianas in the borderlands: Shifting
identities and imagined communities''. In: \emph{Gender \& Society}
22.5, pp.~639--659.

Kunz, Rahel (2008). ``Remittances are Beautiful? Gender implications of
the new global remittances trend''. In: \emph{Third World Quarterly}
29.7, pp.~1389--1409.

Mahler, Sarah J and Patricia R Pessar (2006). ``Gender matters:
Ethnographers bring gender from the periphery toward the core of
migration studies''. In: \emph{International migration review} 40.1,
pp.~27--63.

Massey, Douglas S, Mary J Fischer, and Chiara Capoferro (2006).
``International migration and gender in Latin America: A comparative
analysis''. In: \emph{International migration} 44.5, pp.~63--91.

Semyonov, Moshe and Anastasia Gorodzeisky (2005). ``Labor Migration,
Remittances and Household Income: A Comparison between Filipino and
Filipina Overseas Workers 1''. In: \emph{International Migration Review}
39.1, pp.~45--68.

Curran, Sara R and Estela Rivero-Fuentes (2003). ``Engendering migrant
networks: The case of Mexican migration''. In: \emph{Demography} 40.2,
pp.~289--307.

Pessar, Patricia R and Sarah J Mahler (2003). ``Transnational migration:
Bringing gender in''. In: \emph{International migration review} 37.3,
pp.~812--846.

Cerrutti, Marcela and Douglas S Massey (2001). ``On the auspices of
female migration from Mexico to the United States''. In:
\emph{Demography} 38.2, pp.~187--200.

Hondagneu-Sotelo, Pierrette (2000). ``Feminism and migration''. In:
\emph{The Annals of the American Academy of Political and Social Science}
571.1, pp.~107--120.

Kanaiaupuni, Shawn Malia (2000). ``Reframing the migration question: An
analysis of men, women, and gender in Mexico''. In: \emph{Social forces}
78.4, pp.~1311--1347.

Pedraza, Silvia (1991). ``Women and migration: The social consequences
of gender''. In: \emph{Annual review of sociology} 17.1, pp.~303--325.

\hypertarget{children}{%
\subsection{\texorpdfstring{\emph{Children}}{Children}}\label{children}}

Donato, Katharine M and Blake Sisk (2015). ``Children's migration to the
United States from Mexico and Central America: Evidence from the Mexican
and Latin American migration projects''. In:
\emph{Journal on Migration and Human Security} 3.1, pp.~58--79.

Dreby, Joanna (2012). ``The burden of deportation on children in Mexican
immigrant families''. In: \emph{Journal of Marriage and Family} 74.4,
pp.~829--845.

Donato, Katharine M and Ebony M Duncan (2011). ``Migration, social
networks, and child health in Mexican families''. In:
\emph{Journal of Marriage and Family} 73.4, pp.~713--728.

Huijsmans, Roy (2011). ``Child migration and questions of agency''. In:
\emph{Development and Change} 42.5, pp.~1307--1321.

White, Allen, Caitríona Ní Laoire, Naomi Tyrrell, and Fina
Carpena-Méndez (2011). ``Children's roles in transnational migration''.
In: \emph{Journal of ethnic and Migration Studies} 37.8, pp.~1159--1170.

Dobson, Madeleine E (2009). ``Unpacking children in migration
research''. In: \emph{Children's Geographies} 7.3, pp.~355--360.

\hypertarget{the-family}{%
\subsection{\texorpdfstring{\emph{The
family}}{The family}}\label{the-family}}

Lacroix, Julie and Jonathan Zufferey (2019). ``A Life Course Approach to
Immigrants’ Relocation: Linking Long-and Short-istance Mobility
Sequences''. In: \emph{Migration letters} 16.2, pp.~283--300.

Yeung, Wei-Jun Jean and Xiaorong Gu (2016). ``Left behind by parents in
China: Internal migration and adolescents’ well-being''. In:
\emph{Marriage \& family review} 52.1-2, pp.~127--161.

Nobles, Jenna (2013). ``Migration and father absence: Shifting family
structure in Mexico''. In: \emph{Demography} 50.4, pp.~1303--1314.

-------- (2011). ``Parenting from abroad: Migration, nonresident father
involvement, and children's education in Mexico''. In:
\emph{Journal of Marriage and Family} 73.4, pp.~729--746.

De Haas, Hein and Aleida Van Rooij (2010). ``Migration as emancipation?
The impact of internal and international migration on the position of
women left behind in rural Morocco''. In:
\emph{Oxford development studies} 38.1, pp.~43--62.

Gao, Yang, Li Ping Li, Jean Hee Kim, Nathan Congdon, Joseph Lau, and
Sian Griffiths (2010). ``The impact of parental migration on health
status and health behaviours among left behind adolescent school
children in China''. In: \emph{BMC public health} 10.1, p.~56.

Shauman, Kimberlee A (2010). ``Gender asymmetry in family migration:
Occupational inequality or interspousal comparative advantage?'' In:
\emph{Journal of Marriage and Family} 72.2, pp.~375--392.

Cooke, Thomas J, Paul Boyle, Kenneth Couch, and Peteke Feijten (2009).
``A longitudinal analysis of family migration and the gender gap in
earnings in the United States and Great Britain''. In: \emph{Demography}
46.1, pp.~147--167.

Cooke, Thomas J (2008). ``Migration in a family way''. In:
\emph{Population, Space and Place} 14.4, pp.~255--265.

De Jong, Gordon F and Deborah Roempke Graefe (2008a). ``Family life
course transitions and the economic consequences of internal
migration''. In: \emph{Population, Space and Place} 14.4, pp.~267--282.

-------- (2008b). ``Family life course transitions and the economic
consequences of internal migration''. In:
\emph{Population, Space and Place} 14.4, pp.~267--282.

Desai, Sonalde and Manjistha Banerji (2008). ``Negotiated identities:
Male migration and left-behind wives in India''. In:
\emph{Journal of Population Research} 25.3, pp.~337--355.

Kulu, Hill and Nadja Milewski (2007). ``Family change and migration in
the life course: An introduction''. In: \emph{Demographic research} 17,
pp.~567--590.

Shauman, Kimberlee A and Mary C Noonan (2007). ``Family migration and
labor force outcomes: Sex differences in occupational context''. In:
\emph{Social Forces} 85.4, pp.~1735--1764.

Raghuram, Parvati (2004). ``The difference that skills make: gender,
family migration strategies and regulated labour markets''. In:
\emph{Journal of ethnic and migration studies} 30.2, pp.~303--321.

De Snyder, V Neily Salgado (1993). ``Family life across the border:
Mexican wives left behind''. In:
\emph{Hispanic Journal of Behavioral Sciences} 15.3, pp.~391--401.

Bielby, William T and Denise D Bielby (1992). ``I will follow him:
Family ties, gender-role beliefs, and reluctance to relocate for a
better job''. In: \emph{American Journal of Sociology} 97.5,
pp.~1241--1267.

Boyd, Monica (1989). ``Family and personal networks in international
migration: recent developments and new agendas''. In:
\emph{International migration review} 23.3, pp.~638--670.

Mincer, Jacob (1978). ``Family migration decisions''. In:
\emph{Journal of political Economy} 86.5, pp.~749--773.

\hypertarget{networks}{%
\subsection{\texorpdfstring{\emph{Networks}}{Networks}}\label{networks}}

Creighton, Mathew J and Fernando Riosmena (2013). ``Migration and the
Gendered Origin of Migrant Networks Among Couples in Mexico''. In:
\emph{Social science quarterly} 94.1, pp.~79--99.

Liu, Mao-Mei (2013). ``Migrant networks and international migration:
Testing weak ties''. In: \emph{Demography} 50.4, pp.~1243--1277.

Massey, Douglas S and Maria Aysa-Lastra (2011). ``Social capital and
international migration from Latin America''. In:
\emph{International journal of population research} 2011.

Garip, Filiz (2008). ``Social capital and migration: How do similar
resources lead to divergent outcomes?'' In: \emph{Demography} 45.3,
pp.~591--617.

Mouw, Ted (2006). ``Estimating the causal effect of social capital: A
review of recent research''. In: \emph{Annu. Rev. Sociol.} 32,
pp.~79--102.

Fussell, Elizabeth and Douglas S Massey (2004). ``The limits to
cumulative causation: International migration from Mexican urban
areas''. In: \emph{Demography} 41.1, pp.~151--171.

Aguilera, Michael B and Douglas S Massey (2003). ``Social capital and
the wages of Mexican migrants: New hypotheses and tests''. In:
\emph{Social forces} 82.2, pp.~671--701.

Curran, Sara R and Estela Rivero-Fuentes (2003). ``Engendering migrant
networks: The case of Mexican migration''. In: \emph{Demography} 40.2,
pp.~289--307.

Davis, Benjamin, Guy Stecklov, and Paul Winters (2002). ``Domestic and
international migration from rural Mexico: Disaggregating the effects of
network structure and composition''. In: \emph{Population Studies} 56.3,
pp.~291--309.

Deléchat, Corinne (2001). ``International migration dynamics: The role
of experience and social networks''. In: \emph{Labour} 15.3,
pp.~457--486.

Palloni, Alberto, Douglas S Massey, Miguel Ceballos, Kristin Espinosa,
and Michael Spittel (2001). ``Social capital and international
migration: A test using information on family networks''. In:
\emph{American Journal of Sociology} 106.5, pp.~1262--1298.

Kanaiaupuni, Shawn Malia (2000). ``Reframing the migration question: An
analysis of men, women, and gender in Mexico''. In: \emph{Social forces}
78.4, pp.~1311--1347.

Portes, Alejandro (1998). ``Social capital: Its origins and applications
in modern sociology''. In: \emph{Annual review of sociology} 24.1,
pp.~1--24.

Wilson, Tamar Diana (1998). ``Weak ties, strong ties: Network principles
in Mexican migration''. In: \emph{Human Organization}, pp.~394--403.

Massey, Douglas S, Joaquin Arango, Graeme Hugo, Ali Kouaouci, Adela
Pellegrino, and J Edward Taylor (1993). ``Theories of international
migration: A review and appraisal''. In:
\emph{Population and development review}, pp.~431--466.

Montgomery, James D (1991). ``Social networks and labor-market outcomes:
Toward an economic analysis''. In: \emph{The American economic review}
81.5, pp.~1408--1418.

Granovetter, Mark S (1973). ``The Strength of Weak Ties''. In:
\emph{American Journal of Sociology} 78.6, pp.~1360--1380.

\hypertarget{mexico-us-migration}{%
\section{\texorpdfstring{\textbf{Mexico-US
migration}}{Mexico-US migration}}\label{mexico-us-migration}}

See syllabus for `'Teorias sociales de la migracion''. Also need to talk
about return migration and circular migration.

Durand, Jorge (2017).
\emph{Historia minima de la migracion Mexico-Estados Unidos}. El Colegio
de Mexico AC.

Hill, Kenneth and Rebeca Wong (2005). ``Mexico--US migration: Views from
both sides of the border''. In: \emph{Population and Development Review}
31.1, pp.~1--18.

Fussell, Elizabeth (2004). `\texttt{Sources\ of\ Mexico}s migration
stream: Rural, urban, and border migrants to the United States''. In:
\emph{Social Forces} 82.3, pp.~937--967.

Davis, Benjamin, Guy Stecklov, and Paul Winters (2002). ``Domestic and
international migration from rural Mexico: Disaggregating the effects of
network structure and composition''. In: \emph{Population Studies} 56.3,
pp.~291--309.

Massey, Douglas S and Kristin E Espinosa (1997). `\texttt{What}s driving
Mexico-US migration? A theoretical, empirical, and policy analysis''.
In: \emph{American journal of sociology} 102.4, pp.~939--999.

Roberts, Kenneth D (1997). `\texttt{China}s “tidal wave” of migrant
labor: What can we learn from Mexican undocumented migration to the
United States?'' In: \emph{International Migration Review} 31.2,
pp.~249--293.

\end{document}
